\documentclass{beamer}
%Arquivo com os principais pacotes usados e suas descrições.

%%%%%%%%%%%%%%%%%%%%%%%%%%%%%%%%%%%%%%%%%
% 			Idiomas e Acentos			%
%%%%%%%%%%%%%%%%%%%%%%%%%%%%%%%%%%%%%%%%%
\usepackage[brazil]{babel} % Habilita o uso do idioma português do brasil (PT-BR).
\usepackage[T1]{fontenc} 
%\usepackage{fontspec} % Habilita maior variedade de acentos. Pode ser necessario adicionar outros pacotes.
\usepackage{lmodern} % Habilita o uso da font Latin Modern.


%%%%%%%%%%%%%%%%%%%%%%%%%%%%%%%%%%%%%%%%%
% 				TABELAS					%
%%%%%%%%%%%%%%%%%%%%%%%%%%%%%%%%%%%%%%%%%
\usepackage{tabulary} % Cria tabelas mais facilmente.
\usepackage{booktabs} % Melhora o visual das tabelas.

%\usepackage[table]{xcolor} % Pacote de cor pra as tabelas.

\usepackage{caption} % Melhora as legendas de imagens, tabela etc.

%%%%%%%%%%%%%%%%%%%%%%%%%%%%%%%%%%%%%%%%%
% 				IMAGENS					%
%%%%%%%%%%%%%%%%%%%%%%%%%%%%%%%%%%%%%%%%%
%\usepackage{graphicx} % Facilita a inserção de imagens.


%%%%%%%%%%%%%%%%%%%%%%%%%%%%%%%%%%%%%%%%%
% 			CÓDIGO FONTE				%
%%%%%%%%%%%%%%%%%%%%%%%%%%%%%%%%%%%%%%%%%
%Documentação de código fonte.
\usepackage{listings}


%%%%%%%%%%%%%%%%%%%%%%%%%%%%%%%%%%%%%%%%%
% 	Símbolos e Caracteres Matemáticos	%
%%%%%%%%%%%%%%%%%%%%%%%%%%%%%%%%%%%%%%%%%
\usepackage{amsmath}
\usepackage{amssymb}
\usepackage{amsfonts}
%\usepackage{mathspec} %Habilita o uso das fontes e dos caracteres matematicos.


%%%%%%%%%%%%%%%%%%%%%%%%%%%%%%%%%%%%%%%%%
%				ABNT					%
%%%%%%%%%%%%%%%%%%%%%%%%%%%%%%%%%%%%%%%%%
%\usepackage[alf]{abntcite} % Ordena as referencias em ordem alfabética.
\usepackage{url} %Facilita o uso de url. Pode-se usar o comando \url{...}.


%%%%%%%%%%%%%%%%%%%%%%%%%%%%%%%%%%%%%%%%%
% 			Configurações				%
%%%%%%%%%%%%%%%%%%%%%%%%%%%%%%%%%%%%%%%%%
\captionsetup{justification=centering,labelfont=bf} %Formata a legenda das figuras.
%\graphicspath{{../imgs/}} %Define o diretorio padrão para buscar as imagens da apresentação.  
%\setromanfont[Ligatures=TeX]{Crimson}
%\defaultfontfeatures{Scale=MatchLowercase, Mapping=tex-tex}

%%%%%%%%%%%%%%%%%%%%%%%%%%%%%%%%%%%%%%%%%
%				BEAMER					%
%%%%%%%%%%%%%%%%%%%%%%%%%%%%%%%%%%%%%%%%%
%Define algumas configurações que serão validas para todo o documento.  
\setbeamertemplate{section in toc}[sections numbered]
\setbeamertemplate{subsection in toc}[subsections numbered]
\setbeamertemplate{background canvas}[vertical shading][bottom=blue!3,top=blue!7]
\setbeamertemplate{caption}[numbered]

\PassOptionsToPackage{table}{xcolor}
\usetheme{Berlin}

\author{Guilherme A. de Macedo \\
        Victor H. Carlquist da Silva
}

\title{Agile Modeling}
\date{\today}

\begin{document}

    \frame{
        \titlepage
    }
	
	%victor
	\frame{\frametitle {Introdução}
	    \begin{itemize}
	        \item Modelar e documentar \textit{software};
	        \item Coleção de valores, princípios e práticas;
	        \item Desenvolver os processos com confiança;
	        \item É combinado com outras técnicas (Scrum, ...)
	    \end{itemize}
	}
	%Guilherme
	\frame{\frametitle {Agile Model Driven Development (AMDD)}
	    \begin{itemize}
	        \item Model Driven Development (MDD);
	        \item Agile Model Driven Development (AMDD).
	    \end{itemize}
	}
	%victor
	\frame{\frametitle {Melhores práticas da Modelagem Ágil}
	    \begin{enumerate}
	        \item Outubro de 2000 começou a definição dos valores, princípios e práticas.
	        \item Definição muito específica dos passos, difícil de entender;
	        \item Em 2003 - Agile Model Driven Development (AMDD) (práticas).
	    \end{enumerate}
	}
	%Guilherme
	\frame{\frametitle {Valores da Modelagem Ágil}
	    \begin{itemize}
	        \item Quatro valores herdados da Extreme Programming (Ken Beck):
                \begin{enumerate}
                    \item Comunicação - Modelos promovem a comunicação;
                    \item Simplicidade - É mais fácil entender diagramas do que linhas de códigos;
                    \item Feedback - Comunicar ideias através de diagramas gera feedback é mais rápido;
                    \item Coragem - Para tomar importantes decisões e mudar a direção caso alguma decisão tomada se mostre inadequada.
                \end{enumerate}
            \item Humildade/Respeito - Os melhores desenvolvedores sabem que não sabem tudo.
            \begin{itemize}
                \item Huet Landry sugeriu o conceito de "Outra-Estima" em detrimento da "Auto-Estima".
            \end{itemize}
	    \end{itemize}
	}
	
	%victor
	\frame{\frametitle {Princípios da Modelagem Ágil}
	    \begin{itemize}
	        \item Modele com um propósito;
	        \item Maximizar os investimentos dos Stakeholder;
	        \item Viajar leve;
	        \item Multiplos modelos;
	        \item Feedback rápido;
	        \item Ser simples;
	        \item Aceite Mudanças;
	        \item Mudança incremental;
	        \item Trabalho de qualidade;
	        \item Software funcional é o objetivo primário;
	        \item Outro recursos é o objetivo secundário;
	    \end{itemize}
	}
	%Guilherme
	\frame{\frametitle {Práticas da Modelagem Ágil}
	    \begin{itemize}
	        \item Principais:
	        \item Suplementares:
	    \end{itemize}
	}

	%Guilherme
	\frame{\frametitle {Práticas da Modelagem Ágil - Principais}
	    \begin{itemize}
	        \item Participação ativa dos envolvidos;
	        \item Não modele sozinho;
	        \item Aplique os artefatos corretamente;
	        \item Itere para outro artefato;
	        \item Prove a modelagem com código;
	        \item Use ferramentas simples;
	        \item Modele utilizando incrementos pequenos;
	        \item Armazene as informações em apenas um lugar;
	        \item Propriedade coletiva;
	        \item Trabalhe com vários modelos em paralelo;
	        \item Crie conteúdo simples;
	        \item Descreva modelos simplistas;
	        \item Exponha os modelos publicamente.
	    \end{itemize}
	}
	
	%Guilherme
	\frame{\frametitle {Práticas da Modelagem Ágil - Suplementares}
	    \begin{itemize}
	        \item Aplique padrões de modelagem;
	        \item Tenha cuidado ao aplicar padrões;
	        \item Descarte modelos temporários;
	        \item Formalize os modelos de contrato;
	        \item Atualize somente quando for necessário.
	    \end{itemize}
	}	
	
	%victor & Guilherme
	\frame{\frametitle {Conclusão}
	    \begin{itemize}
	        \item ...
	    \end{itemize}
	}
\end{document}
		
