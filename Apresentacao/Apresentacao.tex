\documentclass{beamer}
\input{Base}
\PassOptionsToPackage{table}{xcolor}
\usetheme{Berlin}

\author{Guilherme A. de Macedo \\
        Victor H. Carlquist da Silva
}

\title{Agile Modeling}
\date{\today}

\begin{document}

    \frame{
        \titlepage
    }
	
	%victor
	\frame{\frametitle {Introdução}
	    \begin{itemize}
	        \item Modelar e documentar \textit{software};
	        \item Coleção de valores, princípios e práticas;
	        \item Desenvolver os processos com confiança;
	        \item É combinado com outras técnicas (Scrum, XP, etc...)
	    \end{itemize}
	}
	%Guilherme
	\frame{\frametitle {Agile Model Driven Development (AMDD)}
	    \begin{itemize}
	        \item Model Driven Development (MDD);
	        \item Agile Model Driven Development (AMDD).
	    \end{itemize}
	}
	%victor
	\frame{\frametitle {Melhores práticas da Modelagem Ágil}
	    \begin{enumerate}
	        \item Outubro de 2000 começou a definição dos valores, princípios e práticas.
	        \item Definição muito específica dos passos, difícil de entender;
	        \item Em 2003 - Agile Model Driven Development (AMDD) (práticas).
	    \end{enumerate}
	}
	%Guilherme
	\frame{\frametitle {Valores da Modelagem Ágil}
	    \begin{itemize}
	        \item Quatro valores herdados da Extreme Programming (Ken Beck):
                \begin{enumerate}
                    \item Comunicação - Modelos promovem a comunicação;
                    \item Simplicidade - É mais fácil entender diagramas do que linhas de códigos;
                    \item Feedback - Comunicar ideias através de diagramas gera feedback é mais rápido;
                    \item Coragem - Para tomar importantes decisões e mudar a direção caso alguma decisão tomada se mostre inadequada.
                \end{enumerate}
            \item Humildade/Respeito - Os melhores desenvolvedores sabem que não sabem tudo.
            \begin{itemize}
                \item Huet Landry sugeriu o conceito de "Outra-Estima" em detrimento da "Auto-Estima".
            \end{itemize}
	    \end{itemize}
	}
	
	%victor
	%
	%
	%
	%
	%
	%portoimagem.wordpress.com
	%
	%
	\frame{\frametitle {Princípios da Modelagem Ágil}
	    \begin{itemize}
	        \item Modelagem Ágil possível 11 princípios;
	    \end{itemize}
	}
	%victor
	\frame{\frametitle {Princípios da Modelagem Ágil}
	\begin{figure}
		\caption{Modele com um propósito}
	        \includegraphics[width = 7cm,keepaspectratio]{img/alvo}
	    \caption*{fonte: coracoesabrasados.blogspot.com}
	    \end{figure}
	}
	%victor
	\frame{\frametitle {Princípios da Modelagem Ágil}
	\begin{figure}
		\caption{Maximizar os investimentos dos Stakeholder}
	        \includegraphics[width = 7cm,keepaspectratio]{img/invetimento}
	    \caption*{fonte: blogdaprata-pb.blogspot.com}
	    \end{figure}
	}
	%victor
	\frame{\frametitle {Princípios da Modelagem Ágil}
	\begin{figure}
		\caption{Viajar leve}
	        \includegraphics[width = 7cm,keepaspectratio]{img/viajar}
	    \caption*{fonte: www.dailytech.com}
	    \end{figure}
	}
	%victor
	\frame{\frametitle {Princípios da Modelagem Ágil}
	\begin{figure}
		\caption{Multiplos modelos}
	        \includegraphics[width = 7cm,keepaspectratio]{img/modelos}
	    \caption*{fonte: www.wikihow.com}
	    \end{figure}
	}
	%victor
	\frame{\frametitle {Princípios da Modelagem Ágil}
	\begin{figure}
		\caption{Feedback rápido}
	        \includegraphics[width = 7cm,keepaspectratio]{img/feedback}
	    \caption*{fonte: beatriziolanda.com}
	    \end{figure}
	}
	%victor
	\frame{\frametitle {Princípios da Modelagem Ágil}
	\begin{figure}
		\caption{Ser simples}
	        \includegraphics[width = 7cm,keepaspectratio]{img/kiss}
	    \caption*{fonte: reginagiannetti.wordpress.com}
	    \end{figure}
	}
	%victor
	\frame{\frametitle {Princípios da Modelagem Ágil}
	\begin{figure}
		\caption{Aceite Mudanças}
	        \includegraphics[width = 7cm,keepaspectratio]{img/mudanca}
	    \caption*{fonte: suzyqscraps.com}
	    \end{figure}
	}
	%victor
	\frame{\frametitle {Princípios da Modelagem Ágil}
	\begin{figure}
		\caption{Mudança incremental}
	        \includegraphics[width = 7cm,keepaspectratio]{img/incremental}
	    \caption*{fonte: pitacoazul.blogspot.com}
	    \end{figure}
	}
	%victor
	\frame{\frametitle {Princípios da Modelagem Ágil}
	\begin{figure}
		\caption{Trabalho de qualidade}
	        \includegraphics[width = 7cm,keepaspectratio]{img/iss}
	    \caption*{fonte: portoimagem.wordpress.com}
	    \end{figure}
	}
	%victor
	\frame{\frametitle {Princípios da Modelagem Ágil}
	\begin{figure}
		\caption{Software funcional é o objetivo primário}
	        \includegraphics[width = 7cm,keepaspectratio]{img/software}
	    \caption*{fonte: www.profissionaisti.com.br}
	    \end{figure}
	}
	%victor
	\frame{\frametitle {Princípios da Modelagem Ágil}
	\begin{figure}
		\caption{Outros recursos é o objetivo secundário}
	        \includegraphics[width = 7cm,keepaspectratio]{img/recur}
	    \caption*{fonte: forum.imasters.com.br}
	    \end{figure}
	}
	
	%Guilherme
	\frame{\frametitle {Práticas da Modelagem Ágil}
	    \begin{itemize}
	        \item Principais:
	        \item Suplementares:
	    \end{itemize}
	}

	%Guilherme
	\frame{\frametitle {Práticas da Modelagem Ágil - Principais}
	    \begin{itemize}
	        \item Participação ativa dos envolvidos;
	        \item Não modele sozinho;
	        \item Aplique os artefatos corretamente;
	        \item Itere para outro artefato;
	        \item Prove a modelagem com código;
	        \item Use ferramentas simples;
	        \item Modele utilizando incrementos pequenos;
	        \item Armazene as informações em apenas um lugar;
	        \item Propriedade coletiva;
	        \item Trabalhe com vários modelos em paralelo;
	        \item Crie conteúdo simples;
	        \item Descreva modelos simplistas;
	        \item Exponha os modelos publicamente.
	    \end{itemize}
	}
	
	%Guilherme
	\frame{\frametitle {Práticas da Modelagem Ágil - Suplementares}
	    \begin{itemize}
	        \item Aplique padrões de modelagem;
	        \item Tenha cuidado ao aplicar padrões;
	        \item Descarte modelos temporários;
	        \item Formalize os modelos de contrato;
	        \item Atualize somente quando for necessário.
	    \end{itemize}
	}	
	
	%victor & Guilherme
	\frame{\frametitle {Conclusão}
	    \begin{itemize}
	        \item ...
	    \end{itemize}
	}
\end{document}
		
